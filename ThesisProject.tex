\documentclass{article}
\usepackage{enumitem}
\usepackage{fixltx2e}
\usepackage{hyperref}
\usepackage{graphicx}
\usepackage{listings}
\usepackage{color}
\usepackage{amsmath}
\usepackage{textcomp}
\usepackage{subcaption}
\usepackage[top=25truemm,bottom=30truemm,left=25truemm,right=25truemm]{geometry}



\setlength\intextsep{5pt}
\setlength\textfloatsep{5pt}


\def\vector#1{\mbox{\boldmath $#1$}}
\begin{document}

\title{Master's Thesis Outline}
\author{Kansuke Ikehara}
\maketitle

\begin{abstract}
\end{abstract}
\tableofcontents


\section{Introduction}
	\subsection{Complex Networks}
	Almost every scientific and engineering discipline deals with data that come from experimental observations or simulations running on a computer. Traditionally, data have been numerals, which may represent temperature, velocity, or voltage, and methodologies that analyze these numerals have been established for over hundreds of years. 
	
	\textit{Graph} or \textit{network}, which is essentially structured data as opposed to simple numeric values, has been used as a new way to approach real-world problems in  recent decades. Social science, for instance, has witness the power of network analysis with the recent emergence of online social network services such as Facebook, Twitter, LinkedIn, etc.  Social scientists have been solving previously intractable questions in the field with the tremendous amount of  data and advanced graph algorithms with high-performance computers. Biological sciences use networks as a tool to dissect biological, chemical and even ecological processes in order to gain insight into how neurons in our brain interact with each other, how nutrients get broken down into pieces and get re-assembled in another molecular form, and how animals eat or be eaten by other animals. Engineering systems such as the Internet, power grids, water distribution networks, transportation networks, etc. have also been investigated with using network analysis tools for constructing efficient and robust systems. 
	
	
	\subsection{Structural Universality of Networks}
	
	One of the main research themes in Network Science, an interdisciplinary field which studies properties of networks ranging from biological and social networks to engineering system networks, is to find hidden universal patterns in the networks of various categories. A model called \textit{small-world network}, introduced by Watts and Strogatz \cite{watts1998cds}, produces networks having properties of: 1. high density of triangles and 2. low average distance between a pair of nodes in a network. These properties are, surprisingly, often found not only in a single category of networks, but also in the wide range of kinds. Another universal property is \textit{power-law} degree distribution sometimes referred to as \textit{scale-free}, in which one observes a number of nodes with few connections while there is few nodes with many connections to others. This very skewed distribution of degree, connections attached to a node, can be found among diverse sets of networks \cite{Barabasi99emergenceScaling}. 


	\subsection{Structural Diversity of Networks}
	Contrary to prevalent small-world and scale-free properties, each category of networks exhibits unique structural profiles of its own. Milo \textit{et al.} introduced a concept called \textit{network motifs} that is essentially patterns of frequent  sub-graph (motif) in a particular network \cite{Milo_motif}. They have shown each category of network, such as gene regulation (transcription), food webs, electronic circuits, etc., has a distinct pattern of network motifs. Furthermore, Milo \textit{et al.} revealed the existence of superfamilies of networks that  are groups of network categories having the highly convergent motif profiles \cite{Milo_SuperFamily}. Onnela \textit{et al.} constructed a taxonomy of 746 networks  of diverse categories using \textit{Mesoscopic Response Functions}(MRFS) that essentially describes the changes of specific values related to the community structure of a network with respect to a parameter $\xi \in [0,1]$\cite{Onnela_Taxonomy}. Each network has its own MRFS and they calculated distance between networks which is defined as the area of difference between two networks' MRFS. They have successfully identified clusters of similar networks in a dendrogram constructed based on MRFS distance.
	
	\subsection{Searching The Underlying Principles}
Networks are merely a simplified mathematical object and yet, one could gain some insight into a phenomenon and/or process of a studied field just from network structure. 
If there is a diverse set of networks of various kinds and scales, what can we say about inter-domain relationships between network categories? The question is threefold: 1. \textit{What aspects of network structure make a specific category of network different from others?} 2. \textit{Are there any sets of network categories that are inherently indistinguishable from each other based on network structure?} 3. \textit{If two networks of different categories are indistinguishable by network structure, are their mechanisms of underlying processes the same?} An example for question one would be strangeness of social networks in network structure compared to other networks as explained in \cite{WhySocialNetworks}.  For question two, since human's eyes are not suitable for distinguishing large networks, one may resort to using a machine learning algorithm and see when the algorithm makes mistakes, implying that a pair of categories of networks is indistinguishable.  An answer to question three may unveil the influence of an underlying process which generates the network or drives growth of the network.

The contribution of answering those questions comes in two ways: 1. it gives us a general conceptual framework upon which networks are studied across domains. Previous studies have only looked at single category or multiple categories as one category, ignoring the relationships between categories. By studying networks across domains, one could find general theories of networks in a more plausible way. 2. it gives us the knowledge base upon which various network-related algorithms are constructed and tuned properly.  For example, it may be applied to a recommendation engine which utilizes customers' purchase habits as a network.

In this thesis, we study 1200 (the number may change later) of various kinds of networks, ranging from ecological food-web to online social network to digital circuit. We extract network statistics as features of a network, construct a high-dimensional feature space and map each network onto the feature space. We then train a machine learning algorithm called \textit{Random Forest} with the network data and let it classify the networks based on their features. As the category distribution is skewed, we try several sampling strategies and show the effect of each methodology. We construct confusion matrices based upon classification results and proceed to analyze the misclassification. We then conclude with discussion based on several hypotheses and classification results.

\section{Data Sets}
	\subsection{Collection and Conversion}
	\subsection{Category Distribution}

\section{Methods}
	\subsection{Network Features}
		\subsubsection{Clustering Coefficient}
		\subsubsection{Modularity}
		\subsubsection{Mean Geodesic Distance}
		\subsubsection{Degree Assortativity}
		\subsubsection{Network Motifs}

	\subsection{Sampling Methods}
		\subsubsection{Random Over Sampling}
		\subsubsection{SMOTE}
	
	\subsection{Classification}
		\subsubsection{Random Forest Classifier}
		\subsubsection{Confusion Matrix}
		\subsubsection{}
		
	\subsection{}

\section{Analyses}
\subsection{}

\bibliographystyle{ieeetr}
\bibliography{reference} 

\end{document}












 
 