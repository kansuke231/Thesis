\documentclass{article}
\usepackage{enumitem}
\usepackage{fixltx2e}
\usepackage{hyperref}
\usepackage{graphicx}
\usepackage{listings}
\usepackage{color}
\usepackage{amsmath}
\usepackage{textcomp}
\usepackage{subcaption}
\usepackage[top=25truemm,bottom=30truemm,left=25truemm,right=25truemm]{geometry}



\setlength\intextsep{5pt}
\setlength\textfloatsep{5pt}


\def\vector#1{\mbox{\boldmath $#1$}}
\begin{document}

\title{Master's Thesis Outline}
\author{Kansuke Ikehara}
\maketitle

\begin{abstract}

\end{abstract}
\tableofcontents


\section{Introduction}
	\subsection{Complex Networks}
	Almost every scientific and engineering discipline deals with data that come from experimental observations or simulations running on a computer. Traditionally, data have been numerals, such as temperature, velocity, voltage, etc., and methodologies that analyze these numerals have been established for over hundreds of years.
	
	 \textit{Graph} or \textit{network}, which is essentially structured data as opposed to simple numeric values, has been used as a new way to approach real-world problems in  recent decades. Social science, for instance, has witness the power of network analysis with the recent emergence of online social network services such as Facebook, Twitter, LinkedIn, etc.  Social scientists have been solving previously intractable questions in the field with the tremendous amount of  data and advanced graph algorithms with high-performance computers. Biological sciences use networks as a tool to dissect biological, chemical and even ecological processes in order to gain insight into how neurons in our brain interact with each other, how nutrients get broken down into pieces and get re-assembled in another molecular form, and how animals eat or be eaten by other animals. Engineering systems such as the Internet, power grids, water distribution networks, transportation networks, etc. have also been investigated with using network analysis tools for constructing efficient and robust systems. 
	
	As such, various fields of science and engineering have extensively utilized networks as a new tool to analyze phenomena and even construct a system of a particular interest. One of the main research themes in \textit{Network Science}, an interdisciplinary field which studies properties of networks ranging from biological and social networks to engineering system networks, is to find hidden universal patterns in those networks. A model called \textit{small-world network}, introduced by Watts and Strogatz, produces networks having properties of: 1. high density of triangles and 2. low average distance between a pair of nodes in a network. These properties are, surprisingly, often found not only in a single category of networks, but also in the wide range of kinds. Another universal property, that is claimed throughout in many disciplines, is \textit{power-law} degree distribution, in which one observes a number of nodes with few connections while there is few nodes with many connections to others. This very skewed distribution of degree, connections attached to a node, can be found among diverse sets of networks.
	
	\subsection{Inference and Comparison of Network Categories}
	\subsection{Searching The Underlying Principles}


\section{Data Sets}
	\subsection{Collection and Conversion}
	\subsection{Category Distribution}

\section{Methods}
	\subsection{Network Features}
		\subsubsection{Clustering Coefficient}
		\subsubsection{Modularity}
		\subsubsection{Mean Geodesic Distance}
		\subsubsection{Degree Assortativity}
		\subsubsection{Network Motifs}

	\subsection{Sampling Methods}
		\subsubsection{Random Over Sampling}
		\subsubsection{SMOTE}
	
	\subsection{Classification}
		\subsubsection{Random Forest Classifier}
		\subsubsection{Confusion Matrix}
		\subsubsection{}
		
	\subsection{}

\section{Analyses}
\subsection{}

\end{document}












 
 