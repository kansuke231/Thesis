\section*{Future Work}
There is still some room for our study to be improved. The class imbalance problem, even though we have utilized sampling methods in order to alleviate the problem, is one of the main remaining concerns. We could possibly discover other hidden properties and relationships if more instances were added for sub-domains that are excluded from the analyses due to the lack of instances needed for classification tasks, such as language network, collaboration network, power grid network, etc. Another direction for future research is incorporating other scale-invariant structural features. In this study we have only used a set of eight features. It is possible, however, that adding other dimensions in the feature space may reveal other hidden properties that were not captured in our feature set. 

Lastly, our study could extend a work by Middendorf \textit{et al.} in which they trained a machine learning classifier on instances of various network generative models, classified protein interaction networks of the drosophila melanogasterusing, a species of fly, and identified a generative mechanism that was most likely to produce the protein interaction networks \cite{MechanismInference}. Using a broad spectrum of networks of different categories, our study suggests a way to construct and validate a hypothesis regarding which network sub-domains have the common generative process. For example, one could train a classifier on only networks that are constructed by some network generative models, feed the classifier instances of real-world networks and observe which categories of networks are classified as which generative processes.
