% ****** Start of file apssamp.tex ******
%
%   This file is part of the APS files in the REVTeX 4.1 distribution.
%   Version 4.1r of REVTeX, August 2010
%
%   Copyright (c) 2009, 2010 The American Physical Society.
%
%   See the REVTeX 4 README file for restrictions and more information.
%
% TeX'ing this file requires that you have AMS-LaTeX 2.0 installed
% as well as the rest of the prerequisites for REVTeX 4.1
%
% See the REVTeX 4 README file
% It also requires running BibTeX. The commands are as follows:
%
%  1)  latex apssamp.tex
%  2)  bibtex apssamp
%  3)  latex apssamp.tex
%  4)  latex apssamp.tex
%
\documentclass[%
 reprint,
%superscriptaddress,
%groupedaddress,
%unsortedaddress,
%runinaddress,
%frontmatterverbose, 
%preprint,
%showpacs,preprintnumbers,
%nofootinbib,
%nobibnotes,
%bibnotes,
 amsmath,amssymb,
 aps,
%pra,
%prb,
%rmp,
%prstab,
%prstper,
%floatfix,
]{revtex4-1}

%\usepackage{savesym} % refer to this: http://www.tex.ac.uk/FAQ-alreadydef.html
\usepackage{longtable}
\usepackage{graphicx}% Include figure files
\usepackage{dcolumn}% Align table columns on decimal point
\usepackage{bm}% bold math

\usepackage{enumitem}
\usepackage{fixltx2e}
\usepackage{hyperref}
\usepackage{listings}
\usepackage{color}
\usepackage[caption=false]{subfig}

%\linenumbers\relax % Commence numbering lines

%\usepackage[showframe,%Uncomment any one of the following lines to test 
%%scale=0.7, marginratio={1:1, 2:3}, ignoreall,% default settings
%%text={7in,10in},centering,
%%margin=1.5in,
%%total={6.5in,8.75in}, top=1.2in, left=0.9in, includefoot,
%%height=10in,a5paper,hmargin={3cm,0.8in},
%]{geometry}

\begin{document}

\preprint{APS/123-QED}

\title{The Structure of Complex Networks across Domains}% Force line breaks with \\
\thanks{A footnote to the article title}%

\author{Kansuke Ikehara}
% \altaffiliation[Also at ]{Department of Computer Science, University of Colorado, Boulder}%Lines break automatically or can be forced with \\
 \email{kansuke.ikehara@colorado.edu}
\author{Aaron Clauset}%
 \email{aaron.clauset@colorado.edu}
 
\affiliation{%
Department of Computer Science, University of Colorado, Boulder
 %This line break forced with \textbackslash\textbackslash
}%

%\date{\today}% It is always \today, today,
             %  but any date may be explicitly specified

\begin{abstract}
The structure of complex networks has been of interest in many scientific and engineering disciplines over the decades. A number of studies in the field have been focused on finding the common properties among different kinds of networks such as heavy-tail degree distribution, small-worldness and modular structure and they have tried to establish a theory of structural universality in complex networks. However, there is no comprehensive study of network structure across a diverse set of domains in order to explain the structural diversity we observe in the real-world networks. In this paper, we study 986 real-world networks of diverse domains ranging from ecological food webs to online social networks along with 575 networks generated from four popular network models. Our study utilizes a number of  machine learning techniques such as random forest and confusion matrix in order to show the relationships among network domains in terms of network structure. Our results indicate that there are some partitions of network categories in which networks are hard to distinguish based purely on network structure. We have found that these partitions of network categories tend to have similar underlying functions, constraints and/or generative mechanisms of networks even though networks in the same partition have different origins, e.g., biological processes, results of engineering by human being, etc. This suggests that the origin of a network, whether it's biological, technological or social, may not necessarily be a decisive factor of the formation of similar network structure. Our findings shed light on the possible direction along which we could uncover the hidden principles for the structural diversity of complex networks.
\end{abstract}

\maketitle

%\tableofcontents


\input chapter1.tex
\input chapter2.tex
\input chapter3.tex
\input chapter4.tex
\input chapter5.tex
\input chapter6.tex
\input chapter7.tex
\input chapter8.tex

\bibliographystyle{apsrev4-1}	% or "siam", or "alpha", etc.
\nocite{*}
\bibliography{reference}		% Bib database in "refs.bib"

\end{document}
%
% ****** End of file apssamp.tex ******