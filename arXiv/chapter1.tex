\section*{Back Ground}%\subsection*{Complex Networks}
Almost every scientific and engineering discipline deals with data that comes from some sort of experimental observations. Traditionally, such data are expressed as numbers, which may represent temperature, velocity, or voltage, and methodologies that analyze the data have been established for over hundreds of years. A \textit{graph} or \textit{network}, which is a kind of data representation describing the relations among some entities such as persons, molecules, animals, logical gates, etc., has been used as a new way to approach, interpret and solve real-world problems in the last decades. Social science, for instance, has witnessed the power of network analysis with the recent emergence of online social network services such as Facebook, Twitter, LinkedIn, etc.\cite{Kleinberg:1}. Previously invisible social phenomena at the scale of off-line social networks, which usually consist of tens to hundreds of nodes at most, have been observed in large scale online social networks with the abundance of online data, faster and more efficient computational resources along with advancements of graph algorithms.
	
	
	Biological sciences use networks as a tool to dissect biological, chemical and ecological processes in order to gain insight into the functionality of such processes. These include brain that consists of networks of neurons \cite{BrainNetwork}, complex metabolic reactions within a human's cells and relationships between malfunctioning metabolic processes and human diseases \cite{MetabolicNetworkAndDiseases} and the effect in biodiversity, that might result from perturbation in an ecological network, such as food-web, mutualistic network, etc \cite{EcologicalNetwork}. Engineering systems such as the Internet \cite{Internet}, power grids \cite{PowerGrid}, water distribution networks \cite{WaterDistribution}, transportation networks \cite{Train}, etc. have also been investigated using network analysis tools for constructing more efficient and robust systems. 
	
	The term \textit{complex network} depicts an essential difference from ordinary graphs having some kind of regular structure that have been studied in the field of mathematics for a long time: the structure of real-world networks almost always exhibits an unusual pattern that greatly deviates from the regular structure and this seemingly irregular and complex structure can often be a clue to the underlying mechanism of the process of interest.  For example, the unusual density of triangles in social networks implies an underlying mechanism in our society: the formation of our social circles tends to be made by local interactions, such as introducing your friend to another friend in your circle that results in a new connection between your friends, making a triangle in the circle. There have been a number of studies that have investigated the structure of complex networks of diverse fields and connections between the structure and the underlying mechanism of the process \cite{Strogatz2001, Newman03thestructure, StatisticalMechanics, boccaletti06}. In the followings subsections we discuss the theories in the structure of complex networks in great detail.
	
	%\subsection*{Common Properties in The Structure of Networks}
	The structure of a network can be characterized in a number of ways, but there are three structural properties that are found to be common in many types of networks: the skewed degree distribution, small-worldness and community structure.
	
	 The degree of a node is a measure of how many edges are connected to a particular node, and the probability distribution $p(k)$ over all nodes essentially describes the "unfairness" in the network. If all of the nodes in a network have the exact same number of connections, the fairest case, then $p(k)$ behaves like Kronecker delta, where $p(k)$ has a value $1$ only at a specific degree $k$. Or if it is less fair, one may observe a narrow Gaussian-like distribution around an average degree. In the real world network, however, this almost never happens. What is observed most of the time instead, is a very skewed degree distribution in which most of the nodes have a few connections incident upon them, and very few nodes have a disproportionally large number of connections with them. \textit{Power-law} distribution is one of candidate distributions that describes the observed phenomenon well, and a network having this distribution is often called \textit{scale-free} network \cite{Barabasi99emergenceScaling}. Surprisingly, a number of networks  from diverse fields, such the Internet, metabolic reactions, World-Wide-Web, etc. have been claimed to be a scale-free network. However, one needs to be careful in order to validate if a network of interest indeed has the power-law degree distribution and it appears that a number of such claims need more statistically valid treatment such one proposed by Clauset \textit{et al.} for justification \cite{Clauset:PowerLaw}. The disproportionality of the skewed degree distribution indicates an existence of  \textit{hubs}, \textit{cores} or \textit{elites} in a network. Such important nodes in a network can play a critical role for a network to function properly and the failure of such nodes may result in a catastrophe \cite{ScaleFreeAttack}.
	
	The next common structural characteristic of networks is \textit{small-worldness}, first conceptually introduced by a Milgram's experiment \cite{Milgram} and mathematically proposed in a paper by Watts and Strogatz \cite{watts1998cds}. Watts and Strogatz proposed a random network model that, depending on a parameter setting, produces a network which has the following properties: (i) the high density of triangles, implying that if three nodes are connected, it is likely that those three nodes actually compose a triangle; (ii) low average distance between a pair of nodes, which indicates that from any node it is just a few steps, in average, to reach any other node in the network. These properties are accomplished by an existence of "long-range" connections bridging together pairs of nodes topologically far away from each other. Small-worldness holds seemingly contradicting properties together: the large degree of local-ness exhibited as a large clustering coefficient value and the large degree of global-ness expressed as a small value of mean geodesic distance.  These properties make small-world networks a very efficient system for information flow \cite{SmallWorldEfficiency} and synchronizing coupled oscillators in the network \cite{SmallWorldSynchronization}. Note that the small-worldness itself is orthogonal to the skewed degree distribution, meaning that so-called small-world networks can be constructed without having the heavy-tail degree distribution. The definition of small world network presented above is, however, not generally used. Most of the researchers today regard small-world property as simply the low average pairwise distance between nodes that grows approximately as $O(\log n)$.
	
	
	Many of the real-world networks contain \textit{modules} or \textit{communities} in which nodes are densely connected but among which there are sparse edges running. This structure is called \textit{community structure} of a network and there has been a number of studies investigating such communities and inventing a new algorithm for detecting communities in the network \cite{Modularity1, Modularity2, ModularityReview}. The communities in a network found by algorithms often correspond to the functional units of the network: a unit of chemical reactions producing vital chemical product in the metabolic network, a group of densely connected neurons taking charge of a cognitive function, such as language and visual processing, and a group of scientists working together in the same field.
	
	
	%\subsection*{Distinguishing Structural Features of Networks}
The properties such as skewed-degree distribution, small-worldness and community structure are found in networks of various kinds, but they alone cannot explain a diversity of networks in terms of network structure. It has been believed by a number of researchers that some classes of network have a set of distinguishing structural features that makes the specific network class "stand out" among others. In this subsection we show some examples of network class that have distinguishing structural features, including social networks, brain networks and subway networks.

Social networks, regardless of whether they are off-line as we see in our daily lives or online like Facebook, have long been known for having  distinguishing structural features: clustering and positive degree assortativity \cite{AssortativeMixing,WhySocialNetworks, Mislove:2007:OnlineSocial}. The large degree of clustering indicates the high probability of one's friends being friends to each other and the positive degree assortativity shows a tendency that high-degree nodes (nodes having many connections) connect to other high-degree nodes while low-degree nodes connect to other low-degree nodes. Real-world networks, \textit{except} social networks, in general exhibit low clustering and negative degree assortativity that are almost in accordance with ones of their randomized counterparts or null models having the same degree distribution. As Newman pointed out in his paper \cite{WhySocialNetworks}, negative degree assortativity is a natural state for networks. Therefore, in order for a network to have positive degree assortativity, it needs a specific structure that favors the assortativity. The community structure of social networks, as mentioned in the paper, is the key aspect as to why they exhibit the properties, namely clustering and positive degree assortativity. People or nodes in social networks usually belong to some sort of groups or communities and people in the same community are likely to know each other. This community membership yields the high clustering in the social network.  For positive degree assortativity, the size of the communities may play an essential role: individuals in a small group can only have low degree whereas individuals in a large group can, potentially, have much larger degree. This approximate correspondence of degree in the groups is essentially degree assortativity itself.


Brain networks, often referred to as \textit{connectome}, have been in the focus of neuroscience in the last decade in order to solve the long-standing scientific question: \textit{How does a brain work?}  The field that studies extensively brain networks using a tool set from mathematics, computer science, etc., is called \textit{connectomics} and one of the recent research topics in this field is investigating how a topology of a brain network affects the brain's function \cite{ComparativeConnectome}. A number of studies show the topological feature of brain networks, including not only human brain but also other animals, is that: (i) they are highly modularly structured; (ii) they have topological connections that are anatomically long, which comes with the high cost of wiring, and make a brain network topologically small, like a small-world network; and 3. they have a core of highly connected nodes, called \textit{rich-club} in some literatures, that connect modules across the network together. The modular structure in a brain network is found to be correlating with a discrete cognitive function of a brain such as processing visual signals from eyes and audio signals from ears. The existence of topologically short yet anatomically long connections enables a brain to have an efficient way to process the information flowing on the network. Furthermore the rich club of a brain network plays an important role of integrating information that is produced and processed in different parts of the brain and this integration of information enables animals including human beings to do complex tasks \cite{BrainEconomy,Crossley09072013}.

Networks of subways in the major cities around the world have an interesting common feature unique to them, that is \textit{core-branch} structure \cite{Train}. They have a ring-shape connections of stations and dense connections therein, referred to as the "core" of the network. In the core, stations are relatively densely connected to each other, enabling residents of a city to move around quickly. From the ring of the core, branches radiate outward connecting stations far from the center of the city. This structural feature is the result of balance between an efficiency of flow of people and cost for constructing rail lines between stations \cite{SpatioalNetwork1,SpatioalNetwork2}. 


	
	%\subsection*{Comparison and Classification of Networks}
	 In the last decades, researchers from the wide range of fields including biology, social science and physics, have been interested in if there is any structural difference between different classes of networks, if there is a set of structural features unique to a specific network class and if there is any "family" of networks in which networks of different classes share the same structural patterns. These questions are usually converted into a problem of comparing and classifying networks according to some criteria and these criteria span from a simple feature such as clustering coefficient, to more complicated ones such as network motifs, which is going to be explained later.
	
	One of the earliest works in network comparison and classification was done by Faust and Skvoretz \cite{Faust.Skvoretz2002Comparing}, in which they compared various kinds of offline social networks, such as grooming relationships among monkeys, co-sponsorship among U.S. Senate of 1973-1974 and so on. Their comparison of networks is based on a statistical model that incorporates parameters each describing a characteristic network structural feature, for example the frequency of a cyclic triangle (here the networks are \textit{directed}, meaning that edges have directionality). The statistical model essentially predicts an existence of an edge $(i,j)$ in a network based on the parameters and the authors used such a statistical model "trained" on a network for predicting an edge in a different network. Their assumption is that if two networks are similar, a model trained on one of the pair should predict well an edge existence in another network. With this assumption they define the Euclidian distance as a function of a summation over all edge predictions, construct a distance matrix and project it onto a two dimensional space using a technique called Correspondence Analysis, which is similar to Principal Component Analysis. They have found the key insight that what makes networks similar in terms of structure is the property of of edges, namely a kind of relation. For example, networks describing agonistic relations, regardless of kinds of species, exhibit the similar network structure. Although this study was pioneering in graph comparison and classification, it only focused on offline social networks, which themselves are a very narrow field of study.
	
	The breakthrough in graph comparison and classification came along a series of papers by Milo \textit{et al.} that introduced the idea of \textit{network motifs} and "super family" of networks \cite{Milo_motif, Milo_SuperFamily}. Network motifs are essentially patterns of frequent sub-graph in a network compared to its randomized networks having the same degree distribution\cite{Milo_motif}. They have shown that each category of network, such as gene regulation (transcription), food webs, electronic circuits, etc., has distinct network motifs. In many cases the distinction of patterns indicates the functional difference in those networks. Furthermore, they revealed the existence of super families of networks that are groups of network categories having the highly convergent motif profiles. These studies are, as far as we know, the first which investigated a diverse set of networks from different domains and found the underlying similarities among the network categories. Nevertheless, this study is far from proving to be a general theory as it only investigated 35 networks of few categories. 
	
	One of the most recent studies in graph comparison and classification investigated 746 networks and constructed a taxonomy of networks \cite{Onnela_Taxonomy}.  Onnela \textit{et al.} used a technique called \textit{mesoscopic response functions} (MRFS) that essentially describes the change of a specific functional value related to the community structure of a network with respect to a parameter $\xi \in [0,1]$. Each network has its own MRFS and the authors calculated the distance between networks which is defined as an area of difference between two networks' MRFS. Their framework successfully identified groups of networks that are similar in terms of community structure. The drawback of study is, however, the fact that the metric they have used for clustering the networks, namely modularity, is implicitly correlated with the size of a network that is a very strong distinguishing feature for classification of networks.

	
	
	%\subsection*{Structural Diversity of Networks}
	% From our hypothesis, groups of networks are prefdiction that understainfding among or between the groups, then we could understand fundamental concepts that creates the diffetence of netowrks;; underlying concept that differentiates the classes of netowrks (<<<= explain this first) (--> this is more of a conclusion stuff )Write down that  function is the key concept that differatiates the classs of networks.
	As we have seen, a number of studies have been conducted in order to find groups or super families of networks that have similar structure. However, only few of them have investigated the fundamental concepts that create the structural differences of networks and none of them have done so with a comprehensive set of complex networks. With the abundance of available network data of various kinds and scales and the techniques from the field of machine learning, we could tackle the fundamental, yet unexplored question: \textit{What creates the structural difference among the groups of networks? } Or in general, \textit{what makes complex networks structurally diverse?} 
	
	 There has been a number of studies trying to discover or formulate an idea that explains the structural \textit{universality} of networks, including the skewed degree distribution, small-world networks, community structure and so on. There is, however, no general theory that explains the structural \textit{diversity} of complex networks across a number of domains/fields. The aim of  this thesis is to establish such a general theory explaining the structural diversity of complex networks. Below are the three  questions we have formulated as research objectives:

\begin{enumerate}
	\item \textit{What aspects of network structure do make a specific category of network different from others?}
	\item \textit{Are there any sets of network categories that are inherently indistinguishable from each other based on network structure?} 
	\item \textit{If two networks of different categories are indistinguishable by network structure, are their mechanisms of underlying processes the same? And vice versa.}
\end{enumerate}

The first question is to, in some extent, an extension to the studies investigating social networks' distinguishing structural features. For example, what kinds of network structure are distinguishing, say metabolic networks from the other kinds? As far as we know, few previous studies have done extensive investigation in finding distinctive characteristics of specific kinds of networks compared to other kinds except social networks.

The second question asks if there is any structural similarity between different kinds of networks. We, however, use the word "indistinguishable" in stead of "similar" since we try to observe the commonality from a confusion matrix of a classifier, where a misclassified instance is considered to be indistinguishable from a class it was classified as because it is so similar to other instances of the wrong class, the algorithm fails to label the instance correctly.


The last question is essentially all about elucidating the meta-structure among the network domains. What if two very distinct domains of networks, say biological and technological ones, exhibit very similar network structure and a classifier misclassifies them many times? Is this because their underlying network generative processes, or their processes on networks themselves are the same? Answering this question helps us understand a mechanism for the formation of a specific network structure.


The contribution of answering those questions comes in two ways: first it gives us a general conceptual framework upon which networks are studied across domains. Previous studies have only looked at a single category or multiple categories as one category, ignoring the relationships between categories. By studying networks across domains, one could find general theories of networks or test a hypothesis across domains in a more plausible way.  For instance, one could test validity of a network model across all of domains and find out which domains the network model can explain well; second it gives us the knowledge base upon which various network-related algorithms could be constructed and tuned properly.  Many practical graph algorithms take no assumption in domain-specific network structure. It may be possible, however, to construct a new algorithm which runs faster and performs more efficiently on a specific kind of networks by taking into account of such domain-specific knowledge. For example, if there exists any unique network structural property of recommendation networks,  it may be applied to construct or fine-tune a recommendation engine that utilizes knowledge of unique network structure. As the size of some real-world networks has grown to an unprecedented scale, domain-specific knowledge in network structure may be a key to analyze such large networks in a faster and more efficient way.

	%\subsection*{Project Description}
In this paper, we study 986 of various kinds of real-world networks, ranging from ecological food-web to online social network to digital circuit along with 575 of synthetic networks generated from four different models. We extract network statistics as features of a network, construct a high-dimensional feature space, where each axis corresponds to one of the network features, and map each network onto the feature space. We then train a machine learning algorithm called \textit{random forest} with the training set of network data in order to learn the function that relates structural features of networks to class labels, namely network domains and sub-domains. As the category distribution is skewed, meaning that some categories of networks have larger number of instances than the other minority categories, we try several sampling methods and show the effect of each methodology. We construct confusion matrices based upon classification results and proceed to analyze the misclassification with which we can answer research questions explained in the previous subsection. We then conclude with discussion based on several hypotheses and experimental results along with same ideas for future work.
