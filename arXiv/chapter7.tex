\section*{Conclusion}
 In this thesis we have studied 986 real-world networks along with 575 synthesized networks in order to formulate a hypothesis about the structural diversity of complex networks across various domains and sub-domains.
 
  Our study successfully identified the distinguishing features for some network sub-domains including, metabolic, ecological food web, online social and communication networks and found out some of these features could naturally explain the process in the network of interest. There are sub-domains such as protein interactions and connectomes, however, that seem to be indistinguishable from others with a set of features we have utilized 
  
   Using machine learning techniques such as random forest classifier, confusion matrix and network community detection algorithm, we have found that there are some categories of networks that are hard to be distinguished from each other by a classifier based solely on their structural features and these groups of structurally similar yet categorically different networks in fact seem to have some common properties, such as the same functionality, physical constraints and generative process of the networks. There are, however, some categories of networks that are found to be structurally similar, and yet our hypothesis seems to lack some theoretical basis for explaining the observed phenomenon.
 
 Nevertheless, our study sheds light on the direction to which we could uncover the underlying principles of network structure: the functionality, constraint and growing mechanism of network may play an important role for the construction of networks having certain structural properties.
 
 
 There is still some room for our study to be improved. The class imbalance problem, even though we have utilized sampling methods in order to alleviate the problem, is one of the main remaining concerns. We could possibly discover other hidden properties and relationships if more instances were added for sub-domains that are excluded from the analyses due to the lack of instances needed for classification tasks, such as language network, collaboration network, power grid network, etc. Another direction for future research is incorporating other scale-invariant structural features. In this study we have only used a set of eight features. It is possible, however, that adding other dimensions in the feature space may reveal other hidden properties that were not captured in our feature set. 

Lastly, our study could extend a work by Middendorf \textit{et al.} in which they trained a machine learning classifier on instances of various network generative models, classified protein interaction networks of the drosophila melanogasterusing, a species of fly, and identified a generative mechanism that was most likely to produce the protein interaction networks \cite{MechanismInference}. Using a broad spectrum of networks of different categories, our study suggests a way to construct and validate a hypothesis regarding which network sub-domains have the common generative process. For example, one could train a classifier on only networks that are constructed by some network generative models, feed the classifier instances of real-world networks and observe which categories of networks are classified as which generative processes.
