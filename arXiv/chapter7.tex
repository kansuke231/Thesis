\section*{Conclusion}
 In this thesis we have studied 986 real-world networks along with 575 synthesized networks in order to formulate a hypothesis about the structural diversity of complex networks across various domains and sub-domains.
 
  Our study successfully identified the distinguishing features for some network sub-domains including, metabolic, ecological food web, online social and communication networks and found out some of these features could naturally explain the process in the network of interest. There are sub-domains such as protein interactions and connectomes, however, that seem to be indistinguishable from others with a set of features we have utilized 
  
  % Method -> Analysis -> findings, but I have method and findings only...
  Using machine learning techniques such as random forest classifier, confusion matrix and network community detection algorithm, we have found that there are some categories of networks that are hard to be distinguished from each other by a classifier based solely on their structural features and these groups of structurally similar yet categorically different networks in fact seem to have some common properties, such as the same functionality, physical constraints and generative process of the networks. There are, however, some categories of networks that are found to be structurally similar, and yet our hypothesis seems to lack some theoretical basis for explaining the observed phenomenon.
 
 Nevertheless, our study sheds light on the direction to which we could uncover the underlying principles of network structure: the functionality, constraint and growing mechanism of network may play an important role for the construction of networks having certain structural properties.
 