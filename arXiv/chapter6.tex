\chapter{Discussion}
In the previous chapter, we have found the distinguishing features for various network sub-domains with possible explanations for the underlying processes of networks, and the hidden similarities among network domains and sub-domains based on subdomain networks we have constructed. In this chapter we are going to synthesize our findings and the previous studies together.

In the investigation of distinguishing features and the separability of sub-domains, we have observed that some network sub-domains are hard to be separated in the high dimensional feature space, which can be seen in the AUC score. This finding has an interesting connection with the subdomain networks we have constructed later: the sub-domains having high separability, such as online social networks and ecological food-webs, tend to be at the \textit{periphery} of the subdomain networks, whereas the subdomains having low separability, such as protein interaction networks and connectome, tend to be at the \textit{core} of the subdomain networks. In the subdomain networks, the core of nodes essentially depicts the sub-domains that are frequently misclassified by a classifier due to their structural similarity and the periphery displays sub-domains that are dissimilar to other sub-domains in terms of network structure. Therefore, it is relatively reasonable to infer that the sub-domains at the core that were not studied for finding the distinguishing features, such as bayesian networks and web graphs, may also exhibit the low separability in the feature space. 


 As we have seen in confusion matrices in Fig.\ref{confusion}, network domains such as Biological, Social, etc. are quite separable in the feature space. Also, confusion matrices of network sub-domains exhibit the strong diagonal patterns except for a few such as bayesian network and web graphs, indicating the separability of networks at the sub-domain level. The separability of networks in the different levels (domain and sub-domain) tells us something about where networks of different types occur within a manifold in the high dimensional feature space we have constructed: in the feature space at the domain level, points that correspond to individual networks in the same network domain occupy some space in an intricately shaped manifold; diagonal elements in the confusion matrices for network domain imply that instances of different network domains occupy different locations within the manifold with some overlap, which could be observed in off-diagonal elements in the confusion matrices. At the network sub-domain level, we see a somewhat different outcome. Some sub-domains in a network domain occupy regions in the feature space that are completely separated. They correspond to the non-overlapping sections of the manifold at the level of network domain. They include, for example, food webs for Biological, peer to peer for Informational, communication for Technological and roads for Transportation. Some network sub-domains, however, almost completely overlap in the manifold with other sub-domains and they correspond to the overlapping sections in the manifold at the network domain level. They include bayesian and web graphs for Informational and water distribution and software dependency for Technological. 


This idea of thinking networks as data points in a manifold of complex shape within some feature space has been explored previously. Corominas-Murtra \textit{et al.} have studied the idea of \textit{Hierarchy} in which each axis corresponds to some feature related to the structure of networks, such as tree-ness, feed-forwardness and orderability \cite{Hierarchy}. In this study, it is shown that different kinds of networks, such as technological, language and neural networks occupy some regions in a feature space. One may notice in this study that some regions in the feature space are not occupied at all by any networks. This observation yields a question about the feature space: \textit{Are some regions of a feature space theoretically possible for networks to occupy?} This question may be answered with the study done by Ugander \textit{et al.} \cite{Ugander:2013}. They have studied a feature space in which each axis corresponds to the subgraph frequency of online social networks and mathematically proved that there are some regions in the feature space that are mathematically infeasible to be occupied. In other words, it is theoretically impossible for networks to have a structural property which corresponds to the region in the feature space. Interestingly, they observed that the real world networks, in this case Facebook networks, only occupy some sections of the theoretically feasible region. Taken together, these studies suggest that networks in the high dimensional feature occupy some regions of the entire possible space that is theoretically feasible. This phenomenon may be due to the fact that the space which is not occupied, yet theoretically feasible region, corresponds to an inefficient structure of a network. Many of the biological networks and technological networks are optimized for a functioning by either natural selection over the course of evolution or effort of designing by a number of engineers and they may push the networks into a certain region in the feature space. The convergence toward a certain region in the feature space seems to happen in both kinds of networks, namely biological and technological networks. This is supported by one of our findings that fungal networks, a kind of biological network developed by a biological process, and water distribution networks, a kind of network designed by engineers, are found to be structurally similar based on the results of confusion matrices. This finding may be due to the fact that their optimization is essentially for efficient flow on the network and cost-reduction of wiring in the networks.
 